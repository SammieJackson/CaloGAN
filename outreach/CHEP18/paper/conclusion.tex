\section{Conclusion and future outlook}\label{conclusion}

The research proves that Generative Adversarial Networks are a good candidate for fast simulation of high granularity detectors typically studied for the next generation accelerators. We have successfully generated images of shower energy deposition with a condition on the particle parameters such as the momentum and the coordinates using modern generative deep neural network techniques such as Wasserstain GAN with gradient penalty.

Future work will be focused on improving reproducing second order cluster characteristics, sach as variations and long range corellations between different cells.

The research leading to these results has received funding from Russian Science Foundation under grant ``Applications of probabilistic artificial neural generative models to development of digital twin technology for Non-linear stochastic systems'' under agreement No 19-71-30020. 